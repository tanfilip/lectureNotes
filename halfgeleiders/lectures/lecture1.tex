\section{Basics}
\subsection*{Crysal Structure}
\textbf{Material Structure:}
\begin{itemize}
	\item \textbf{Lattice} = Periodische arrangement of atoms in a crystal.
	\item \underline{in a crystal:} atom never strays far from a single, fixed position.
	\item \textit{\textbf{Kristallijn}:} Deeltjes bestaat uit kristal(len) met een zekere \textit{kristalstructuur} of gebrek daaraan.
	\begin{itemize}
		\item \underline{Monokristallijn:} eenvoudig continue kristallijn.
		\item \underline{Multikristallijn:} $mm$ tot $cm$ korrelgrootte.
		\item \underline{Polykristallijn:} $\mu m$ korrelgrootte.
	\end{itemize}
	\item \textbf{Amorf:} geen periodiciteit, geen ordening.
	\item \textbf{Unit Cell}: groep van deeltjes dat een hele lattice kan vormen.
	\begin{itemize}
		\item \textbf{Primitive unit cell}: kleinste cell.
		\item \textbf{Conventional unit cell}: gekozen voor conventie, los gedefinieerd.
		\item \textbf{Parameters:}
		\begin{itemize}
			\item Lattice constante: $\overrightarrow{a}$, $\overrightarrow{b}$, $\overrightarrow{c}$.
			\item \textbf{Lattice punt $\overrightarrow{R}$} = $h \overrightarrow{a} + n \overrightarrow{b} + p \overrightarrow{c}$, met integers $h$, $k$, $l$.
		\end{itemize}
	\end{itemize}
	\item \textbf{Miller Indices:} Notatie in \textit{crystallography} voor lattice richtingen en vlakken.
	\begin{itemize}
		\item \textbf{punt:} $(h,k,l)$.
		\item \textbf{Richting:} $[hkl]$.
		\item \textbf{parallelle richting:} $<hkl>$.
		\item \textbf{vlak:} $(hkl)$.
		\item \textbf{parallelle vlak:} $\{hkl\}$.
		\item \textbf{negative nummer:} $\overline{|nummer|}$.
	\end{itemize}
	\textbf{TO DO}
\end{itemize}
\subsection*{Band Structure of a Material}
The interactions between 2 identical atoms, including \textit{attraction, repulsion} between atoms, cause a \textbf{shift in the energy levels}. Instead of 2 levels, $N$ separate \& closely spaced levels are formed.
\\*$\Rightarrow$ When $N$ is large $\rightarrow$ \textbf{Continuous Band of Energy.} \\* \\*
\textbf{Conductieband:} \textit{De Hogere band}
\begin{itemize}
	\item[=] Lowest range of \textit{vacant electronic states}.
	\item Empty at $O$ Kelvin temperature.
\end{itemize}
\textbf{Valentieband:} \textit{De Lagere band}
\begin{itemize}
	\item[=] \textit{Highest range of electron energy}, more negative.
	\item 100\% filled at $0$ Kelvin temperature.
\end{itemize}
\textbf{Bandgap:}
\begin{itemize}
	\item[=] In semiconductors \& insulators, conduction \& Valence bands are \textbf{separated} by a bandgap.
	\item[=] Energy range in a solid where \textit{no electron states can exist} due to quantization of energy.
\end{itemize}
\begin{table}[h]
	\centering
	\begin{tabular}{|p{0.3\textwidth}|p{0.3\textwidth}|p{0.3\textwidth}|}
		\hline \textbf{\underline{Conductieband}} & \textbf{\underline{Bandgap}} & \textbf{\underline{Valentieband}} \\
		\hline = de hogere band & = energie bereik tussen conductie- \& valentieband  & = de lagere band \\ \hline
		
	\end{tabular}
	\caption{Difference between energy bands.}
	\label{e_bands}
\end{table}

\begin{table}
	\centering
	\begin{tabular}{|p{0.3\textwidth}|p{0.3\textwidth}|p{0.3\textwidth}|}
		\hline \textbf{\underline{Metal}} & \textbf{\underline{Semiconductor}} & \textbf{\underline{Insulator}} \\ \hline
		Partial filled condition band or condition band overlaps valnce band. & valence electrons form \textbf{strong bonds} \\ \hline
		No bandgap & energy gap of order $1eV$ & large bandgap \\ \hline
	\end{tabular}
	\caption{Difference between metal, semiconductor \& insulator.}
	\label{material_type}
\end{table}