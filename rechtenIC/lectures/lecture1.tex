\section{Inleiding tot het Recht}

\subsection*{Het begrip "Recht"}
\textbf{Leven in een samenleving ordenen/organiseren via het "recht":}
\begin{itemize}
	\item \textit{\textbf{Bemiddelende functie}}:
	\begin{itemize}
		\item Oplossing voor \textbf{conflictsituaties}.
		\item (doen) naleven van bepaalde gedragsregels.
		\item  bv. eigendomsrecht, strafrecht.
	\end{itemize}
	\item \textit{\textbf{Ordenende functie}}:
	\begin{itemize}
		\item Ook regeling buiten \textbf{conflictsituaties}.
		\item Maatschappelijle mechanismen vlot doen functioneren.
		\item bv. verkeersreglement (vb. recht v/d wegrijden als regel), indeling v/d rechterlijke macht.
	\end{itemize}
\end{itemize}
$\Rightarrow$ \textbf{\underline{Recht} = geheel van regels voor uiterlijk menselijk gedrag in de samenleving \& georganiseerde ordening daarvan}.
\begin{itemize}
	\item \textbf{Het recht geeft \underline{Normen:} weer} wat mag, moet of verboden is.
	\item Regels over uitwendig gedrag (vs. loutere gedachten) gebaseerd op normen.
	\item Uitgevaardigd door een \textbf{bevoegd} orgaan $\rightarrow$ het Staat.
	\item \textbf{Handhaving \& sancties:} afdwingbaarheid! $\Rightarrow$ Het recht kan $\overline{nt}$ bestaan, als er ook geen sanctiemechanisme ervoor bestaan.
\end{itemize}
\textbf{Kenmerken van het Recht}
\begin{enumerate}
	\item \textbf{Gedragsvoorschriften die maatsch. bepaald zijn:}
	\begin{itemize}
		\item Vari\"eren naar gelang de (evoluerende) samenleving.
		\begin{itemize}
			\item bv. euthanasie. transgenders, homohuwelijk,...
		\end{itemize}
		\item geheel van regels vormt het "rechtssysteem".
	\end{itemize}
	\item Die uitgevaardigd $\overline{w}$ door een bevoegd orgaan (daartoe erkend)
	\begin{itemize}
		\item regels moet gekend \& erkend zijn.
		\item \textbf{Hi\"erarchie} binnen de instanties:
		\item[$\Rightarrow$] regionaal $\rightarrow$ federaal  $\rightarrow$ Supranationaal (EU)  $\rightarrow$ internationaal.
		\end{itemize}
	\item Het bestaan van een sanctie om naleving te verzekeren.
	\begin{itemize}
		\item Uitgaand van een georganiseerd gezag.
		\item op een wijze die de wet voorziet.
		\item vs. regels van de moraal, godsdienst.
	\end{itemize}
\end{enumerate}

\textbf{Objectief vs. subjectief recht:}

\begin{tabular}{|p{0.45\textwidth} | p{0.45\textwidth} |}
	\hline \textbf{\underline{Objectief recht}} & \textbf{\underline{Subjectief recht}} \\
	\hline geheel van regels dat uiterlijke gedraging van mensen in een maatschappij regelt en dat $\overline{w}$ afgedwongen door de overheid ("law|") & o.b.v. (objectief) recht beschermde aan- spraken op andermans gedrag ("right"). $\Rightarrow$ Het recht dat ik die "law" kan claimen.\\ \hline
\end{tabular}

\subsection*{De Bronnen van het Recht (Overzicht)}
\begin{enumerate}
	\item \textbf{Wet}
	
	\item \textbf{Rechtspraak}
	\begin{itemize}
		\item[=] Beslissingen van rechtbanken.
		\begin{itemize}
			\item Interpreteren v/d wet (en andere bronnen)
			\begin{itemize}
				\item bv. auteursrecht
			\end{itemize}
		\item toepassen in concrete gevallen (concrete omstandingheden.)
		\end{itemize}
	\end{itemize}
	\item \textbf{Gewoonte}
	
	\item \textbf{Algemene Rechtsbeginselen}
	\item \textbf{Rechtsleer}
\end{enumerate}
\subsection*{De Wet als Primaire Rechtbron}

\subsection*{De Rechterlijke Macht in Belgi\"e}

\subsection*{Het Subjectief Recht}
