\section{Zingeving \& Geluk binnen een Neoliberaal Tijdsgewricht}

\subsection*{Onze Neoliberale Samenleving}
\begin{itemize}
	\item Zoektocht naar geluk \& zingeving(?)
	\item Doorgedreven economisering v/d publiek \& private sfeer(?)
	\item \textbf{Dirk De Wachter} (psychiater) - \textbf{\textit{Borderline times}}:
\end{itemize}
\begin{center}
	"\textit{Borderline} is een diepgaand patroon van instabiliteit in intermenselijke relaties, zelfbeeld \& affecten en van duidelijke impulsiviteit, beginnen in de vroege volwassenheid en tot uiting komend in diverse situaties."
\end{center}
\begin{itemize}
	\item[$\Rightarrow$] parallel met samenleving.
	\item[$\Rightarrow$] meerdere kenmerken.
\end{itemize}

\begin{enumerate}
	\item \textbf{Verlatingsangst}:
	\begin{itemize}
		\item Eenzaamheid.
		\item Wegvallen sociale verbanden.
	\end{itemize}
	\item\textbf{ Intense \& Instabiele relaties}:
	\begin{itemize}
		\item Relaties \& instabiele relaties.
		\item Wegwerpcultuur met focus op genot $\Rightarrow$ \textbf{Hedonistisch}
	\end{itemize}
	\item \textbf{Agressie}:
	\begin{itemize}
		\item "Zinloos geweld"
		\item Uiting van toenemend wantrouwen (?)
	\end{itemize}
	\item \textbf{Fundamentele/Existenti\"ele Vragen}:
	\begin{itemize}
		\item Verval religieuze/levensbeschouwelijke kaders.
		\item Schijnantwoorden.
		\item Vlucht in genotsmiddelen.
	\end{itemize}
	\item \textbf{Affeclabiliteit}:
	\begin{itemize}
		\item Door gebrek aan stabiliteit.
		\item Falend zelfbeeld.
		\item Genot- \& Kickcultuur centraal (e.g. Binge drinking).
	\end{itemize}
	\item \textbf{Impulsiviteit}:
	\begin{itemize}
		\item Bij het nemen van allerhande beslissingen.
		\item Versterkt door consumptiemaatschappij.
		\item[$\Leftrightarrow$] Langdurig \textit{engagement}.
	\end{itemize}
	\item \textbf{Stressgebonden parano\"ide/Dissociatiesymptomen}:
	\begin{itemize}
		\item Verlies met realiteit.
		\item Gewone $\Leftrightarrow$ virtuele werkelijkheid.
		\item Media bepaald.
	\end{itemize}
	\item \textbf{Automutilatie \& Su\"icidaliteit}:
	\begin{itemize}
		\item Toename van beide fenomenen.
		\item Toegenomen lichaamscultuur.
	\end{itemize}
	\item \textbf{Leegte \& Zinloosheid}:
	\begin{itemize}
		\item Verlies v/d "grote verhalen"
		\item Ontzuiling.
		\item Secularisatie.
	\end{itemize}
\end{enumerate}

\subsection*{Zingeving \& Geluk}
\begin{itemize}
	\item \textit{Wat is geluk?}
	\item \textit{Waar kan men geluk vinden?}
	\item \underline{Handboek:} weg naar geluk volgens Christendom.
	\item[$\Rightarrow$] Enkele richtingswijzers.
	\item[$\Rightarrow$] Gebaseerd op werk van \textbf{Henri Nouwen}
	\item[$\Rightarrow$] De weg naar innerlijke vrijheid.
\end{itemize}

\begin{enumerate}
	\item \textbf{Leven in het \underline{Hier \& Nu}}:
	\begin{itemize}
		\item \textit{LC 12:16-31}
		\item Bewustwording voor het heden.
		\item[$\Rightarrow$] Los van afleiding.
		\item[$\Rightarrow$] Los van bekommernissen verleden \& toekomst.
	\end{itemize}
	\item \textbf{Hoop}:
	\begin{itemize}
		\item \textit{Rom 5:1-5}
		\item \textit{Espoir} vs. \textit{esp \'{e}rance}.
		\item Niet schuwen van pijn, lijden \& onheil.
		\item Vertrouwen op gedragenheid van God.
		\item Geborgenheid.
		\item[$\Rightarrow$] in relaties met naasten.
		\item[$\Rightarrow$] kan lijden veroorzaken.
		\item[$\Leftrightarrow$] eenzaamheid.
	\end{itemize}
	\item \textbf{Liefde}:
	\begin{itemize}
		\item Moderne portraitering vs. realiteit?
		\item \textit{God is liefde}
		\item Liefde = belangrijk.
		\item Voorwaardelijk \& tijdgebonden vs. onvoorwaardelijk (?)
	\end{itemize}
	\item \textbf{Vreugde}:
	\begin{itemize}
		\item \textit{Gal 5:22}
		\item[$\Leftrightarrow$] Verbittering \& defaitisme.
		\item  Zich bemind weten (door God).
		\item Vertrouwen dat het beter zal gaan (parallel met hoop).
	\end{itemize}
	\item \textbf{Gastvrijheid}:
	\begin{itemize}
		\item \textit{Lc 24:13-32}
		\item \underline{Nu:} instrumentalisering v/d ander \&	de planeet.
		\item[$\Rightarrow$] Ego\"isme.
		\item Ruimte \& openheid cre\"eren.
	\end{itemize}
	\item \textbf{Discipline}:
	\begin{itemize}
		\item \textit{1 Kor 9:24-27}
		\item Discipline \& training = Belangrijk.
		\item[$\Rightarrow$] ook in \textbf{spirituele/ morele} leven.
		\item[$\Rightarrow$] Zie bv. Monniken
		\item Fatalisme (vs. Vrijheid?)
		\item Aandachtigheid \& geduld.
	\end{itemize}
\end{enumerate}
