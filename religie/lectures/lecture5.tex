\section{Leven na de Dood}

\begin{table}[h]
	\centering
	\begin{tabular}{| p{0.45\textwidth} | p{0.45\textwidth}|}
		\hline 
		\textbf{\underline{Oosterse religies}} & \textbf{\underline{Westerse Religie}} 
		\\ \hline
		Opgaan in het grotere geheel & Lichaam \& ziel
		\\ \hline
		Geen individualiteit & Individualistisch
		\\ \hline
		\textbf{Re\"incarnatie} & Leven na de dood, \textbf{verrijzenis}
		\\ \hline
	\end{tabular}
\end{table}

\subsection*{Moeulijk te geloven}
\begin{itemize}
	\item Huidig maatschappelijk klimaat:
	\begin{itemize}
		\item Leven in het \textbf{hier \& nu}.
		\item YOLO.
		\item Vermijden \& dood.
	\end{itemize}
	\item \textbf{\textit{Martin Heidegger}:}
	\begin{itemize}
		\item \textit{sein zum Tode} (menselijke conditie).
		\item Vlucht door religie \& techniek.
		\item \textit{Gelassenheit}
		\item Onmogelijk om te geloven in leven na de dood.
	\end{itemize}
\end{itemize}

\subsection*{De zin van de Dood}
\begin{itemize}
	\item \textit{Heeft de dood zin?}
	\begin{itemize}
		\item \textbf{Gedachtenexperiment:} eeuwig leven.
	\end{itemize}
	\item Positieve intu\"ities leven na de dood.
	\begin{itemize}
		\item Liefde ook na de dood.
		\item Ethisch goed handelen moet beloond worden.
		\item Eigen aan elke cultuur.
	\end{itemize}
\end{itemize}
\subsection*{Verrijzenis}
\begin{itemize}
	\item[=] Belangrijkste geloofspunt Christendom.
	\item Meerdere teksten \& beelden:
	\item Lege graf.
	\item Verschijningen.
	\item \textit{"Eeuwig leven"}.
	\item[$\Rightarrow$] Niet letterlijk.
	\item[$\Rightarrow$] Leven in liefde/nabijheid van God?
\end{itemize}

\section{Over het Goede Leven: Ethiek}
\begin{itemize}
	\item Ethiek/Moraal:
	\begin{itemize}
		\item Goede of slecht handelen.
		\item Niet zwart en wit.
		\item \textbf{Contextueel} (fragmentatie v/d ethiek).
		\item Fundamentele vs. meta-ethiek.
	\end{itemize}
\end{itemize}
\subsection*{Levensbeschouwing \& Ethiek}
\begin{itemize}
	\item \textbf{Levensbeschouwing \& religie = beïnvloeden ethisch denken}:
	\begin{itemize}
		\item vb. \textit{Universal Declaration of Human Rights}
		\begin{itemize}
			\item 10 december 1948, \textit{United Nations}.
			\item Resonanties van Christendom.
			\item Moeilijkheid erkenning in sommige landen.
		\end{itemize}
		\item Eenheid \& consistentie? invulling varieert soms.
		\item Sluimerend aanwezigheid.
		\item Komt naar voorgrond in "\textit{crisissituaties}".
		\item Conflicten tussen waarden (resultaat vaak tot compromis).
	\end{itemize}
\end{itemize}
\subsection*{Ethiek in het Christendom}
\begin{itemize}
	\item Dubbelgebod v/d liefde.
	\item 2 bronnen van ethiek:
	\begin{enumerate}
		\item \textbf{Bijbel:}
		\begin{itemize}
			\item \textbf{Anti-fundamentalisme:} Niet letterlijk lezen \& navolgen.
			\item Gids voor persoonsvorming (\textit{zie deugdenethiek}).
		\end{itemize}
		\item \textbf{Traditie \& leergezag}
		\begin{itemize}
			\item Ethische richtlijnen voor \textbf{contemporaine uitdagingen}.
		\end{itemize}
	\end{enumerate}
\end{itemize}