\section{Geloof \& Wetenschap}
\subsection*{Verschillende Modellen}
Examenvraag: link de modellen met de gezegde.
\begin{enumerate}
	\item \textbf{Harmoniemodel}:
	\begin{itemize}
		\item Tijdens \textbf{Middeleeuwen}.
		\item  Gecentrisme was heersend \textbf{kosmische model}
		\item[$\Rightarrow$] geen onderscheid tussen geloof en wetenschap (i.e. \textbf{harmonie})
		\item Kennis van God door:
		\begin{itemize}
			\item Bijbel.
			\item  Natuur.
		\end{itemize}
		\item Andere wereldbeeld \textbf{Geocentrisme}
		\item Causaliteit (oorzaak-gevolg):
		\begin{itemize}
			\item Onbewogen beweger (\textit{Prima Causa})
			\item Engelen (\textit{Causae Secundae})
		\end{itemize}
	\end{itemize}
	\item \textbf{Conflictmodel}
	\begin{itemize}
		\item \textbf{Moderne tijd}
		\item Heliocentrisme (i.p.v. geocentrisme) $\Rightarrow$ Copernicaanse wereldbeeld.
		\begin{itemize}
			\item[=] Het zon staat centraal van de universum.
			\item  \textit{Copernicus (15-16 eeuw)}
			\item[$\Rightarrow$] Deze idee verder verspreid door \textbf{Galileo} (16-17 eeuw).
		\end{itemize}
		\item Dialogo ... $\rightarrow$ lijst van verbannen boeken $\Rightarrow$ Volgens gelovige \textbf{Blasfemie}.
		\item \textbf{De\"isme}:
		\begin{itemize}
			\item analoog met \textit{Refined Watch Maker}.
			\item[$\Rightarrow$] Probeert geloof \& wetenschap samen te denken.
			\item \textbf{Mechanistisch} (i.e. natuurwetten)
			\item[$\Rightarrow$] God bestaat, maar heeft zich teruggetrokken uit de Kosmos/natuur.
			\item Implicaties:
			\begin{itemize}
				\item \textbf{Geen openbaring}: (nog in profeten, Jezus of Schrift)
				\item Religieuze Autoriteit is onnodig
				\item Wonderen zijn \textbf{Irrationeel}.
				\item Geen religie heeft/is DE waarheid.
			\end{itemize}
		\end{itemize}
		\item \textbf{Sci\"entisme} (Auguste Comte)
		\begin{itemize}
			\item Wetenschap zonder religie.
			\item[$\Rightarrow$] \textbf{Mythologisch/religieus} (irrationeel) $\Rightarrow$ Wetenschappelijk of rationeel (meer volwassen).
			\item[$\Rightarrow$] Humane Wetenschapppen gingen zich conformeren aan ideaal van sci\"entisme.
		\end{itemize}
		\item \textbf{Nieuwe Athe\"isme}
		\begin{itemize}
			\item Geloof en God = \textbf{Waanidee\"en}
		\end{itemize}
	\end{itemize}
	\item \textbf{Kloofmodel} (Wittgenstein)
	\begin{itemize}
		\item \underline{Begin}: negatief t.o.v. religie.
		\item \underline{Later}: Religie is een \textbf{apart taalspel}.
		\begin{itemize}
			\item Onmogelijk om te vergelijken wetenschap met religie $\Rightarrow$ = Aparte talen.
			\item \underline{Religie}: probeert WAAROM vragen te beantwoorden.
			\item[$\Rightarrow$] Religie beheerst wat de waarheid is.
			\item \underline{Wetenschap}: probeert HOE vragen te beantwoorden.
		\end{itemize}
		\item Wat is falisifieerbaar? $\rightarrow$ zie wetenschap vs. pseudowetenschap.
		\item Herman De Dijn: 3 vormen van weten. NK!!!!!!
	\end{itemize}
	\item \textbf{Dialoogmodel}: kloof te groot?
	\begin{itemize}
		\item Verzoening van Kloof- en harmoniemodel.
		\item Religie heeft nood aan wetenschap (wetenschappelijke rationaliteit)
		\begin{itemize}
			\item Religie of sekte? $\Rightarrow$ kritisch denken.
			\item Geen eenduidig definitie voor religie of sekte.
		\end{itemize}
		\item Wetenschap heeft nood aan religie:
		\begin{itemize}
			\item Ethische reflectie (kan a.d.h.v. filosofie maar ook van theologische ethiek).
			\item Kritiek op instrumentalistische rationaliteit (Niet alles kan gevat worden op wetenschappelijke termen).
			\item Antwoorden op fundamentele vragen.
		\end{itemize}
		\item Beperktheid wetenschap, e.g. Kuhns theorie omtrente wetensch. revoluties. $\Rightarrow$ \textit{Paradigma shifts.}
		\begin{itemize}
			\item "Alle wetensch. theorie\"en kunnen vervangen/verbeterd worden"
			\item zie bv. Copernicus, die heliocentrisme als nieuwe paradigma dient, i.p.v. geocentrisme,
			\item  nieuwe paradigma moet verdedigd en meerdere keren bevestigd door anderen worden.
		\end{itemize}
		\item Voorwaarden voor Dialoog:
		\begin{itemize}
			\item Respect voor taalspel/eigenheid v/d ander. ()
			\item Erkennen dat er versch. taalspelen zijn.
			\item De resultaten v/h ene mag niet dienstbaar gemaakt worden aan het andere.
		\end{itemize}
	\end{itemize}
\end{enumerate}